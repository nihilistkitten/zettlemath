% !TEX TS-program = lualatex
\documentclass{article}
\author{Riley Shahar}

\pagestyle{plain}
\usepackage[margin=1.3in]{geometry}
\usepackage[parfill]{parskip}

\usepackage{amsfonts, amscd, amssymb, amsthm, amsmath}
\usepackage{
	capt-of,
	csquotes,
	enumerate,
	fancyhdr,
	float,
	mathtools,
	multicol,
	natbib,
	pifont,
	tabularx,
	tikz-cd,
	resources/quiver,
}
\usepackage{color}
\usepackage[
 unicode=true,
 pdfusetitle,
 colorlinks=true,
 linkcolor=blue,
 citecolor=blue,
 urlcolor=blue,
 runcolor=blue,
 final
]{hyperref}
\usepackage{cleveref}
\usepackage[shortlabels]{enumitem}
\setlist[itemize]{noitemsep}
\setlist[enumerate]{noitemsep}

%% THEOREMS %%
\theoremstyle{plain}
\newtheorem{thm}{Theorem}[section]
\newtheorem*{thm*}{Theorem} % no numbering
\newtheorem{lemma}[thm]{Lemma} % use the same counter as theorem
\newtheorem{prop}[thm]{Proposition} % ditto
\newtheorem{cor}{Corollary}[thm] % reset for every theorem
\newtheorem{prob}{Problem}
\newtheorem{prob_lemma}{Lemma}[prob]
\newtheorem{exc}{Exercise}[section]

\theoremstyle{definition}
\newtheorem{dfn}{Definition}[section]
\newtheorem*{dfn*}{Definition}
\newtheorem{ex}{Example}[section]

\theoremstyle{remark}
\newtheorem*{rmk}{Remark}
\newtheorem*{notation}{Notation}

%% ALPHABETS %%
\def\cA{\mathcal{A}}
\def\cB{\mathcal{B}}
\def\cC{\mathcal{C}}
\def\cD{\mathcal{D}}
\def\cE{\mathcal{E}}
\def\cF{\mathcal{F}}
\def\cG{\mathcal{G}}
\def\cH{\mathcal{H}}
\def\cI{\mathcal{I}}
\def\cJ{\mathcal{J}}
\def\cK{\mathcal{K}}
\def\cL{\mathcal{L}}
\def\cM{\mathcal{M}}
\def\cN{\mathcal{N}}
\def\cO{\mathcal{O}}
\def\cP{\mathcal{P}}
\def\cQ{\mathcal{Q}}
\def\cR{\mathcal{R}}
\def\cS{\mathcal{S}}
\def\cT{\mathcal{T}}
\def\cU{\mathcal{U}}
\def\cV{\mathcal{V}}
\def\cW{\mathcal{W}}
\def\cX{\mathcal{X}}
\def\cY{\mathcal{Y}}
\def\cZ{\mathcal{Z}}

\def\AA{\mathbb{A}}
\def\BB{\mathbb{B}}
\def\CC{\mathbb{C}}
\def\DD{\mathbb{D}}
\def\EE{\mathbb{E}}
\def\FF{\mathbb{F}}
\def\GG{\mathbb{G}}
\def\HH{\mathbb{H}}
\def\II{\mathbb{I}}
\def\JJ{\mathbb{J}}
\def\KK{\mathbb{K}}
\def\LL{\mathbb{L}}
\def\MM{\mathbb{M}}
\def\NN{\mathbb{N}}
\def\OO{\mathbb{O}}
\def\PP{\mathbb{P}}
\def\QQ{\mathbb{Q}}
\def\RR{\mathbb{R}}
\def\SS{\mathbb{S}}
\def\TT{\mathbb{T}}
\def\UU{\mathbb{U}}
\def\VV{\mathbb{V}}
\def\WW{\mathbb{W}}
\def\XX{\mathbb{X}}
\def\YY{\mathbb{Y}}
\def\ZZ{\mathbb{Z}}

%% OTHER SYMBOLS %% 
\DeclareMathOperator{\id}{id}
\DeclareMathOperator{\Hom}{Hom}
\DeclareMathOperator{\Map}{Map}
\DeclareMathOperator{\Obj}{Obj}
\DeclareMathOperator{\dom}{dom}
\DeclareMathOperator{\cod}{cod}

\def\pathto{\rightsquigarrow}

\newcommand{\catname}[1]{{\normalfont\texttt{#1}}}

% https://tex.stackexchange.com/a/118217
\DeclarePairedDelimiter\ceil{\lceil}{\rceil}
\DeclarePairedDelimiter\floor{\lfloor}{\rfloor}

% zettle things
\makeatletter
\newcommand\@zf[2][]{\href{run:#2.pdf}{#1}}
\newcommand\zf{\@dblarg\@zf}
\makeatother

\directlua{tags = {}}
\renewcommand{\tag}[1]{\directlua{table.insert(tags, "#1")}}
\newcommand{\printtags}{\directlua{tex.print(table.concat(tags, ", "))}}

\directlua{scripts = require("scripts")}
\begin{document}


\begin{dfn*}[Category]
	A \emph{category} $\cC$ consists of a collection of \emph{objects} $\Obj(\cC)$
	and a collection of \emph{morphisms} $\Hom(\cC)$, such that

	\begin{itemize}
		\item Each morphism $f$ has a specific \emph{domain} $\dom(f)$ and
		      \emph{codomain} $\cod(f)$; we write $f: X\rightarrow Y$ or
		      $X\xrightarrow{f} Y$.
		\item Each object $X$ has a specific \emph{identity morphism} $1_X:
			      X\rightarrow X$.
		\item For each pair of morphisms $X\xrightarrow{f} Y\xrightarrow{g}
			      Z$, there is a specific \emph{composite morphism} $gf:
			      X\rightarrow Z$.
	\end{itemize}
	This data must be \emph{associative} and \emph{unital}, which is to say,
	\begin{itemize}
		\item (Associativity) For any triplet of composition-compatible morphisms
		      $X\xrightarrow{f}Y \xrightarrow{g}Z\xrightarrow{h}W$, we have $h(gf) =
			      (hg)f$. We write merely $hgf$.
		\item (Unital) For any $f: X\rightarrow Y$, we have $1_Yf = f = f1_X$.
	\end{itemize}

	In other words, the following diagrams commute:

	\begin{figure}[H]
		\centering
		\begin{multicols}{2}
			\begin{tikzcd}
				X & Y & Z & W
				\arrow["f", from=1-1, to=1-2]
				\arrow["g", from=1-2, to=1-3]
				\arrow["h", from=1-3, to=1-4]
				\arrow["gf"', curve={height=18pt}, from=1-1, to=1-3]
				\arrow["hg", curve={height=-18pt}, from=1-2, to=1-4]
			\end{tikzcd}

			\begin{tikzcd}
				X & X \\
				& Y
				\arrow["1_X", from=1-1, to=1-2]
				\arrow["f", from=1-2, to=2-2]
				\arrow["f"', from=1-1, to=2-2]
			\end{tikzcd}
			\begin{tikzcd}
				X & Y \\
				& Y
				\arrow["f", from=1-1, to=1-2]
				\arrow["1_Y", from=1-2, to=2-2]
				\arrow["f"', from=1-1, to=2-2]
			\end{tikzcd}
		\end{multicols}
	\end{figure}
\end{dfn*}

Tags: \printtags

\hypersetup{%
pdfkeywords=\directlua{tex.print("{" .. "\printtags" .. "}")}
}
\end{document}

